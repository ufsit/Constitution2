\documentclass{article}
\usepackage{times}
\usepackage{datetime}
\usepackage[a4paper, total={6in, 8in}]{geometry}

\newdateformat{monthyeardate}{%
  \monthname[\THEMONTH] \THEYEAR}

\title{Cybersecurity Training and Networking \\ Constitution}
\date{\monthyeardate\today}

\usepackage{titlesec}

\titleformat{\section}
  {\normalfont\large\bfseries\MakeUppercase}
  {ARTICLE \Roman{section}:}{0.5em}{}

\titleformat{\subsection}
  {\normalfont\bfseries}
  {Section \Alph{subsection}.}{0.25em}{}
  

\def\clubname{CTN }

\def\clubfullname{Cybersecurity Training and Networking }

\begin{document}

\maketitle

\section{Name of Organization}

The name of this organization is \clubfullname. The organization will use the acronym \clubname in publicity materials and correspondence


\section{Organization Affiliation}

As outlined in the RSO Classification Policy, \clubfullname is considered a General Registered Student Organization. As
a GRSO, \clubfullname is a separate 3rd party entity, not considered a part of the University of Florida. Through
registering with the University, \clubfullname agrees to follow its policies and operate on campus with access and/or
eligibility for specific campus benefits.


\section{Purpose Statement}

The primary purpose of \clubname is to provide members of the UF community with development and networking opportunities in information security. The organization will conduct meetings on a variety of topics related to information security in order to fulfill this purpose.


\section{Compliance Statement}

Upon approval by the Department of Student Engagement, \clubname shall be a registered student organization at the University of Florida. \clubname shall comply with all local, state and federal laws, as well as all University of Florida regulations, policies, and procedures. Such compliance includes but is not limited to the University’s regulations related to Non-Discrimination, Sexual Harassment (including sexual misconduct, dating violence, domestic violence, and stalking), Hazing, Commercial Activity, and Student Leader Eligibility

\subsection{Non-Discrimination}
\clubname agrees that it will not discriminate on the basis of race, creed, color, religion, age, disability, sex, sexual orientation, gender identity and expression, marital status, national origin, political opinions or affiliations, genetic information and veteran status as protected under the Vietnam Era Veterans' Readjustment Assistance Act. Discrimination on the basis of the protected classes described in University of Florida Regulation 1.006 (Non-Discrimination/Harassment/Invasion of Privacy Policies) is prohibited.

\subsection{Sexual Harassment}

\clubname agrees that it will not engage in any activity that is unwelcome conduct of sexual nature that creates a hostile environment. Behaviors that could create a hostile environment include sexual harassment (which could include inappropriate sexual comments), sexual misconduct, dating violence, domestic violence, and stalking and repeated instances of cyber abuse. Sexual harassment as described in University of Florida Regulation 1.006 (Non-Discrimination/Harassment/Invasion of Privacy Policy) is prohibited.

\subsection{Hazing}
\clubname agrees that it will not initiate, support, or encourage any events or situations that recklessly, by design, or intentionally endanger the mental or physical health or safety of a student for any purpose including but not limited to initiation or admission into or affiliation with any student group or organization. Hazing as defined in the University of Florida Regulation 1.0081 (Prohibition of Hazing; Procedures and Penalties) and 4.040 (Student Honor Code and Student Conduct Code) is prohibited. If found responsible for hazing, sanctions may be imposed against the organization, its leaders and/or members.

\subsection{Responsibility to Report}
The University of Florida identifies Responsible Employees and Campus Security Authorities to support the health, safety, and wellbeing of campus. If \clubname becomes aware of any such conduct described in this article, they are encouraged to report it immediately to staff in Student Engagement, the Director of Student Conduct and Conflict Resolution, the University’s Title IX Coordinator, or to their Student Organization Advisor, who are identified as mandated reporters.

\subsection{Officer Eligibility}
\clubname understands, acknowledges, and agrees to uphold and abide by the specific minimal requirements regarding officer eligibility as defined in the Registered Student Organization Classification and Officer Eligibility Policy.


\section{Membership}
Membership in this organization is open to all enrolled students at the University of Florida. Non-enrolled students, spouses, faculty, and staff are prohibited from holding membership, office or voting powers. All members are free to leave and disassociate without fear of retribution, retaliation, or harassment.


\section{Bylaws for \clubname}

\clubname may elect to maintain separate bylaws document to outline the day-to-day operations of the organization and to clarify policies and procedures otherwise not included in the previous articles. Bylaws and/or other guiding documents may not take precedence over the requirements sent forth by local, state, and federal laws, the university of Florida’s regulations, policies, and procedures, and the Student Engagement constitution requirements. Amendments and changes may be made to the bylaws and shall be consistent with the Student Engagement approved constitution on file and student engagement’s constitution requirements. Should the organization transition leadership, all bylaws and guiding documents will be transitioned to new student organization leaders and/or advisor(s). \clubname agrees to provide all unaltered by laws and guiding documents and/or clarify its procedures in writing to any University of Florida student, faculty, or staff upon request.


\section{Student Organization Advisor}
Each registered student organization must have an eligible student organization advisor. The student organization advisor must be a full-time, salaried faculty or staff member not on extended leave for 4 consecutive weeks or longer during their advisor term. The advisor and CISE holds the responsibility to oversee the day-to-day functions and operations of \clubname, including the management of its finances, the selection of its members, and ensuring the organization adheres to University and department policies.

\subsection{Eligibility}
The faculty advisor shall meet all University of Florida requirements, and be committed to furthering the goals of the \clubfullname.

\subsection{Selection of the Faculty Advisor}
The faculty advisor shall be selected by a majority vote of the executive board

\subsection{Duties of the Faculty Advisor}
\begin{itemize}
    \item Provide assistance and guidance to the executive board when requested
    \item Attend and oversee officer elections, if he/she is available
    \item Perform other duties at the request of the executive board
\end{itemize}
\subsection{Terms}
\begin{itemize}
    \item The faculty advisor shall serve until he/she resigns or is removed
    \item If the advisor is not fulfilling their duties to the satisfaction of the current executive board, the officers shall hold a majority vote to remove the faculty advisor

\end{itemize}
\subsection{Replacement of the Faculty Advisor}
In the case that the current faculty advisor leaves or is removed, the next advisor will be chosen by the current executive board with support from the current advisor when appropriate. 
The current advisor, before leaving, should pass on any important information to the newly appointed one. 


\section{Officers}
Registered student organizations are required to have a minimum of a President, Treasurer, and Vice President as elected officers. These officers must abide by the Registered Student Organization Classification and Officer Eligibility Policy.

The elected officers of \clubname shall be President, Vice-President, and Treasurer. At no time should one person hold more than one of these positions

\subsection{Eligibility}
Individuals holding office must meet the University of Florida’s criteria for officers of student organizations.

\subsection{Additional Eligibility Criteria}
Individuals holding office must be considered active student members of the \clubfullname (having attended at least 3 GBMs that semester) as per at least two weeks prior to their election.

\subsection{Officer Positions and Duties}

The executive board shall consist of the following elected officers: President, Vice President, and Treasurer. The board may appoint additional officers, as detailed in the bylaws. \\

\noindent
The duties of the President include:

\begin{itemize}
    \item Supervise, coordinate and exercise executive power over all activities of the organization
    \item Call to order and preside over all executive board and general body meetings of the organization
    \item Ensure that all officers are performing their duties, as defined in this constitution and any by-laws
    \item Create a budget at the beginning of each semester, in conjunction with the Treasurer
    \item Be one of two officers with direct access to the bank account. The President will only access the account funds with the approval of the executive board, or in the absence of the Treasurer
    \item Maintain communication with the Center for Student Activities and Involvement, to ensure that the organization’s information is current and in compliance
    \item Maintain communication with the organization’s faculty advisor
    \item Be responsible for facilitating the resolution of internal organizational conflicts and taking any necessary steps not prohibited by the constitution to ensure the effective operation of the organization
    \item Assign additional duties to other officers, as necessary
    \item Provide all relevant documents and records to the succeeding President 
\end{itemize}

\noindent
The duties of the Vice President include:

\begin{itemize}
    \item Assist the President in the performance of their duties. 
	\item Assume the responsibilities of the President in their absence or inability
    \item Be responsible for facilitating the resolution of internal organizational conflicts in which the President has a clear and demonstrated conflict of interest
    \item Carry out additional duties as assigned by the President
    \item Provide all relevant documents and records to the succeeding Vice President
\end{itemize}

\noindent
The duties of the Treasurer include:

\begin{itemize}
    \item Keep an accurate account of all funds received and expended
    \item Create a budget at the beginning of each semester, in conjunction with the President
    \item Become familiar with the Student Government rules for funding
    \item Liaise with Student Government for funding purposes
    \item Be one of two officers with direct access to the bank account. The Treasurer will be responsible for managing the bank account.
    \item Present the current balance of the bank account at all executive board meetings
    \item Assume the responsibilities of the President in the event of both the President and the Vice President’s absence or inability
    \item Carry out additional duties as assigned by the President
    \item Provide all relevant documents and records to the succeeding Treasurer
\end{itemize}


\subsection{Removal of Officers}

Any elected officer may propose the removal of an officer to the executive board, in the event an officer is failing to fulfill their duties or is willfully acting to the detriment of the organization. An officer will be removed from office by a majority vote of the executive board and the general body, or by a two-thirds vote of the executive board. The general body shall consist of all active student members present at a meeting. The President may appoint an interim officer to serve in the vacated position, until an election may be held.

Replacement of this officer will default to the next person that would have received the role as per the most recent election. If there is nobody to fill the position, there will be a general body election for that position as per the guidelines in Article IX.
% Remember to change this if the article number changes

\subsection{Term of Office}

The term of office for all officer positions is 1 year beginning and ending on the date of election.

\section{Elections}
\subsection{Timeline and Dates of Election}
Regular officer elections shall be held in each spring semester during March or April, with the exact date to be determined by the Executive Board. Notice to the general body must be given at least three weeks prior to the date of election, and there must be at least one meeting in the intervening period between the announcement of the election and the election itself. Special elections may be called by the executive board if an elected position has been vacated (through resignation or the removal process), and must be announced at least two weeks prior.

The election may be helld concurrently with an election for officers of an Eligible Equivalent Organization.

\subsection{Eligibility}
The officer must be considered an active member of \clubname.

\subsection{Nominations}
Nominations may be made by any active member of \clubname. Nominations for each elected position will be taken in one of the following ways:
\begin{itemize}
    \item On the floor of a meeting in the intervening period between the announcement of the election and the election itself;
    \item By correspondence with the Secretary up to 48 hours before the election meeting. The Secretary shall respond to such correspondence to indicate receipt of the nomination; or
    \item On the floor of the elections meeting, immediately preceding voting for the office in question
\end{itemize}

Before the election, the Vice President shall verify the active member status of all individuals who have been nominated to run for office.

\subsection{Ballots}
The Vice President may prepare ballots for the election. If prepared, election ballots shall include all nominations submitted up to 48 hours before the elction and a space to write in the names of candidates not on the ballot, for candidates nominated on the day of the election.

\subsection{Election Procedure}

% Remember to change this if the article listing executive board positions changes
At the meeting during which the election is to take place, the highest-ranking officer of the organization who is not running for office in the current election shall preside over the elections. Blank paper shall be provided for the election by the Secretary for use as ballots.

Election of officers shall occur in the order in which executive board positions are listed in Article VIII, Section C, beginning with the Presidency. Candidate speeches, debate, and/or question and answer opportunities will be provided prior to the elections for each office, the format of which shall be decided by the executive board at the time of setting the election date. Only active student members may pose questions or engage in the discussion.

After debate, secret ballots shall be submitted by all voting members and tabulated by the presiding officer, observed by the Secretary and faculty advisor, if present. Potential officers shall be elected by a majority of votes cast, unless no candidate receives a majority. In this case, a runoff election will occur between the two candidates who received the greatest number of votes. In the event of a tie during the runoff or a tie in the first round that would result in more than two candidates making the runoff, the sitting President will break the tie. Potential officers may slate down to run for a lower office if not elected to the office he/she was initially nominated for. There is no limit on the number of times a candidate may slate down.

\subsection{Length of Term}

The newly elected executive board will take office at a joint meeting of the outgoing and incoming boards, which shall occur no later than the beginning of the Summer semester. Officers will serve until the next such meeting after the following election.


\subsection{Procedures in the case of lack of direction}

In the absence of clear direction on election, amendment, and /or voting procedures, \clubname agrees to follow the guidance and instruction of Robert's Rules of Order for the election or amendment process.


\section{Finance}
As a General Registered Student Organization, \clubfullname does not receive any funding or resources from other UF Departments or Colleges, rather, this organization is funded by Student Government and fundraisers. Active members are expected to participate in fundraising activities. Money raised will be used to fund meetings, t-shirts, and professional development for the club

\subsection{Budget}
The Treasurer, working with the President, shall create a budget for each semester. The executive board and the active student membership must approve the budget within the first two weeks of each relevant semester. Additional expenditures not included in the semester budget must be approved by a majority vote of the executive board.

\subsection{Financial Authority}
Only the President and Treasurer will be authorized signers with the organization’s financial institution. If issued, charge cards will only be authorized to these two officers. The organization’s funds may be spent on activities and supplies as deemed appropriate by the executive board, but shall not be used for anything illegal under University of Florida or government laws and regulations.

\subsection{Transition of Financial Authority}
The outgoing President and Treasurer are responsible for transferring financial account authority to the incoming President and Treasurer within two weeks of the relevant officer election. Outgoing officers may not maintain any financial authority or account access once they have transferred responsibility to their incoming counterparts.

\subsection{Handling of Funds}
Funds, such as membership dues or donations, may be collected by any elected officer, but must be turned over to the Treasurer or President within 48 business hours. Once received by the Treasurer or President, funds shall be deposited within 48 business hours of collection, absent exceptional circumstances.

\subsection{Financial Obligation}
\clubname will not require membership dues; however, it will raise funds through car washes and similar activities, for t-shirts, travel to leadership conferences, and other operational expenses of sign the organization. Members are expected to participate in these fundraising activities. The \clubfullname will also seek funds from external sponsors.

As a GRSO, \clubname will comply with UF Finance and Accounting policies on purchasing, funding and fundraising.


\section{Dissolution of Organization}
Upon dissolution, student organizations are prohibited from leaving their organizational assets to any individual or any other student organization. Rather, student organizations may designate a specific charity that will receive such organizational assets. At the time of dissolution, after all outstanding debts are paid, \clubname will leave any assets and outstanding funds to CISE

\subsection{Dissolution}
In the event that this organization dissolves, all monies left in the account, after outstanding debts and claims have been paid, shall be donated to the University of Florida Association for Computing Machinery student organization.

\section{Amendments to Constitution}
Student Engagement has established a process through which constitutions may be amended, reviewed, and approved. Student organizations wishing to amend their constitutions must utilize their constitution on file listed on GatorConnect to make amendments and submit those changes to Student Engagement.

\subsection{Proposal Procedure}
Amendments shall be proposed by one of the following mechanisms:
\begin{enumerate}
	\item Unianimous agreement of a Constitutional Committee composed of at least 3/4ths of \clubname board members and 5 active student members, or
	\item Unianimous agreement of a Constitutional Committee composed of 1 \clubname board member and at least 2/3rds of the board members from an Eligible Equivalent Organization
\end{enumerate}

Amendments may be proposed by the Constitution Committee to the general body, after the Committee has approved them by a unanimous vote. Additionally, any active student member may submit proposed amendments to the general body subject to any procedures outlined in bylaws.

\subsection{Ratification Procedure}
Amendments shall be ratified by a two-thirds vote of active student members present at the meeting. Amendment votes must be announced at least one week prior. 

All amended constitutions must be submitted directly to the Department of Student Activities and Involvement for review and approval. 

\section{Active Membership}
An student enrolled at UF is considered an active member during a semester if:
\begin{enumerate}
	\item Attended at least 3 meetings hosted, co-hosted, or primarily funded by \clubname, or
	\item Is considered an active member by an Eligible Equivalent Organization.
\end{enumerate}

\section{Eligible Equivalent Organization}
A student organization is considered an ``Eligible Equivalent Organization'' if all of the following requirements are met
\begin{enumerate}
	\item The organization is a Registered Student Organization (RSO) at the University of Florida,
	\item The organization's primary mission is related to computer or information security, 
	\item The organization is considered an extension of the CISE department, and
	\item The organization has continuously existed for at least five consecutive years, including the current year.
\end{enumerate}

\section{Meetings}
\subsection{General Meetings}
The general membership of the organization shall meet at least once per month during the fall and spring semesters, and at least once during the summer semester. The meetings may be held in collaboration with other UF registered student organizations.

\subsection{Officer Meetings}
The executive board of the organization shall meet at least once per month during the fall and spring semesters, and at least once during the summer semester. Officer meetings are open to all active student members.

\subsection{Calling Meetings}
The President is responsible for calling all general and officer meetings. In the absence of the President, the Vice President has the authority to call a meeting. The Director of Communications is responsible for making a good faith effort to inform all members of any meeting. Such notification must be issued at least 72 hours prior to any meeting.

\end{document}
